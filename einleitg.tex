\chapter*{Einleitung}
%Historie des Textes.\\
%\\Berufsbilder/erforderliches Wissen.\\
%Die Berufsbilder des Tontechnikers, Toningenieurs und Tonmeister �berschneiden sich heutzutage in vielen Bereichen. Das Bet�tigungsfeld dieser Tonschaffenden erfordert - je nach spezieller Anforderung - Wissen und Fertigkeiten aus den unterschiedlichsten Bereichen. Insgesamt kann die Besch�ftigung im Tonbereich auf folgenden unterschiedlichen Wissensgebieten und nat�rlich Begabungen aufbauen: Akustik (Schall/Schallausbreitung, Raumakustik, Psychoakustik, musikalische Akustik), Technik (Elektrotechnik, Mikrophone/Schallwandler, Signalverarbeitung, Digitaltechnik), Angewandte Technik ((Bedienung von) Studioger�ten, Mikrophonverfahren/Positionierung), Menschenkenntnis/Psychologie (Aufnahmeleitung, Umgang mit K�nstlern), �berzeugend Reden k�nnen (???), Organisationstalent, Musikkenntnis (Unterst�tzen des musikalischen Ziels bei der Aufnahme, musikalische Nachbearbeitung), musikalische Kreativit�t (ver�nderndes Eingreifen in den musikalischen Proze� durch Arrangement, Effekte, Nachbearbeitung, Ausdruck), Musikwissenschaft (Aufnahmevorbereitung), Differenziertes H�ren (klangliches H�ren, musikalisches H�ren, technisches H�ren).
%\\Welche dieser Bereiche hier.\\
%\\Zielsetzung des Textes.\\
%Dieser Text ist aus einem Skript zum Tutorium in den ersten Semestern des Tonmeisterstudiums gewachsen. Er versucht, Grundlagenwissen zur Ton- und Aufnahmetechnik kurz und anschaulich, aber auf Basis akademischen Anspruchs zu vermitteln. Dieser Anspruch manifestiert sich beispielsweise in �bungsaufgaben am Ende jeden Kapitels, die das Verst�ndnis des Erlernten sicherstellen sollen. 
%\\Absetzung gegen�ber �hnlichen Werken.\\
%Da manche Themengebiete bewu�t nicht in ersch�pfender Ausf�hrlichkeit behandelt werden, befinden sich ebenfalls am Ende jeden Kapitels Literaturverweise, die dem Leser einen einfachen Einstieg zur Vertiefung der entsprechenden Thematik bieten sollen.
%\\Gliederung des Textes.\\
Dieses Skript fa�t einige Grundlagen zusammen, die in den ersten Semestern im Tutorium des Tonmeisterstudiums zu vermitteln sind. Es erhebt keinen Anspruch auf Vollst�ndigkeit, sondern soll zur weiteren Besch�ftigung mit der erh�ltlichen Fachliteratur anregen.

Da diese Unterlagen auch zum sp�teren Nachschlagen dienen k�nnen, ist eine Gliederung in Themenschwerpunkte sinnvoll. Die Abfolge einzelner Punkte im Tutorium kann also unter Umst�nden von der hier vorgenommenen Gliederung abweichen.
