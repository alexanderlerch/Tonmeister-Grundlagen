\chapter{Elektroakustische Wandler} \thispagestyle{empty}

    Der Sinn eines elektroakustischen Wandlers ist die Umwandlung einer physikalischen Gr��e
    des Schallfeldes wie beispielsweise des Schalldruckes in eine elektrische Gr��e, i.a. eine
    Spannung $U$ (Schallempf�nger), oder umgekehrt (Schallsender). Alle in der Tonstudiotechnik
    gebr�uchlichen Wandler besitzen eine Membran, welche durch das Schallfeld in Schwingungen
    versetzt wird, und einen elektrischen Teil, der diese mechanischen Schwingungen in
    elektrische Schwingungen wandelt. Es existieren verschiedene Wandlerprinzipien, die beiden
    in der Tonstudiotechnik i.a. verwendeten sind

    \begin{itemize}
        \item  der \textit{elektrostatische} Wandler ($\rightarrow$
        Kondensatormikrophon) und

        \item  der \textit{elektrodynamische} Wandler ($\rightarrow$
        dynamisches Mikrophon).
    \end{itemize}

    Nachdem nun der Begriff des Wandlers allgemein eingef�hrt ist, soll im folgenden nur noch
    von Mikrophonen gesprochen werden. Lautsprecher als Schallsender funktionieren im
    allgemeinen nach dem Prinzip des elektrodynamischen Wandlers.

    \section{Das Kondensatormikrophon}\label{chap:kondensatormikro}

    \section{Das elektrodynamische Mikrophon}\label{chap:dynmikro}

        Das dynamische Mikrophon besteht grunds�tzlich aus einer mit der Membran verbundenen Spule,
        die sich innerhalb eines Magnetfeldes bewegt. In den Leiterschleifen wird dadurch eine
        Spannung $U_i$ induziert. Nach dem Induktionsgesetz ist diese Spannung

        \begin{equation}\label{eq:uind}
        U_i=B\cdot l\cdot v.
        \end{equation}
        \footnotesize{$B$ ist die St�rke des Magnetfeldes,\\ $l$ ist die L�nge der Leiterschleife,
        die sich in dem Magnetfeld bewegt, und \\ $v$ ist die Geschwindigkeit, mit der sich diese
        bewegt.}\\
